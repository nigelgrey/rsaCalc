% Created 2020-03-01 Sun 10:19
% Intended LaTeX compiler: pdflatex
\documentclass[presentation]{beamer}
\usepackage[utf8]{inputenc}
\usepackage[T1]{fontenc}
\usepackage{graphicx}
\usepackage{grffile}
\usepackage{multimedia}
\usepackage{longtable}
\usepackage{wrapfig}
\usepackage{listings}
\usepackage{rotating}
\usepackage[normalem]{ulem}
\usepackage{amsmath}
\usepackage{textcomp}
\usepackage{amssymb}
\usepackage{capt-of}
\usepackage{hyperref}
\usetheme{default}
\author{Nigel Grey}
\date{\today}
\title{RSA Exploration}
\begin{document}

\maketitle
\begin{frame}{Outline}
  \tableofcontents
\end{frame}
\section{RSA Overview}
\subsection{Introduction}
\begin{frame}{RSA Overview}{Things You May Already Know \ldots}
  \begin{itemize}
    \item The Rivest-Shamir-Adleman algorithm is the most widely used asymmetric
      cryptographic scheme
      \pause
    \item Common applications include:
      \begin{itemize}
        \item Key transport
        \item Digital signatures
      \end{itemize}
      \pause
    \item Central concept: \textbf{Multiplying two large prime numbers is easy,
      but factoring the result is very hard!}
  \end{itemize}
\end{frame}
\subsection{Key Generation}
\begin{frame}{RSA Overview}{Key Generation Procedure}
  \begin{enumerate}
    \item Select two large prime numbers $p$ and $q$
      \pause
    \item Calculate $n=p*q$
      \pause
    \item Calculate $\phi(n)=(p-1)(q-1)$
      \pause
    \item Find $e$ (the ``public exponent'') $e \in \{1,2,\ldots,\phi(n)-1\}
      \mid gcd(e,\phi(n))=1$
      \pause
    \item Calculate the private key $d \mid d*e \equiv 1\mod(\phi(n))$
  \end{enumerate}
\end{frame}
\subsection{Encryption/Decryption}
\begin{frame}{RSA Overview}{Encryption/Decryption}
  \begin{itemize}
    \item \textbf{Encryption:} Given the public key $(n,e)=k_{pub}$ and some plaintext $x$ the RSA
      encryption function is defined as: \\ $y \equiv x^{e}\mod n \mid x,y \in \mathbb{Z}_{n}$
      \pause
    \item \textbf{Decryption:} Given the private key $d=k_{priv}$ and the
      ciphertext $y$ the decryption function is: \\ $x \equiv y^{d} \mod n \mid x,y \in \mathbb{Z}_{n}$
  \end{itemize}
\end{frame}
\section{C Implementation}
\subsection{An Impractical Example}
\begin{frame}{C Implementation}{An Impractical Example}
  \begin{center}
    \movie[ poster, externalviewer]{\includegraphics[width=0.7\textwidth]{image.png}}{cExample.mp4}
  \end{center}
\end{frame}
\begin{frame}{C Implementation}{Problems}
  \begin{itemize}
    \item RSA encryption is deterministic
      \pause
    \item Small values of $e$ and $x$ are vulnerable to attacks if no
      padding is used
      \pause
    \item Plaintext $x=-1,0,1$ map to ciphertext $y=-1,0,1$
      \pause
    \item RSA is malleable (i.e. an attacker could transform the
      ciphertext)
  \end{itemize}
\end{frame}
\section{Go Implementation}
\subsection{Solving Problems}
\begin{frame}{Go Implementation}{Solving Problems}
  \begin{itemize}
    \item Add a padding scheme to solve most of the problems from the
      previous slide i.e. Optimal Asymmetric Encryption Padding (OAEP) as
      defined in the Public Key Cryptography Standard
      \pause
    \item Add a SHA-256 hash function for digital signature
      \pause
    \item Use 2048-bit keys, which should be secure for the next 10
      years or so \ldots
  \end{itemize}
\end{frame}
\subsection{A Practical Example}
\begin{frame}{Go Implementation}{A More Practical Example}
  \begin{center}
    \movie[ poster, externalviewer]{\includegraphics[width=0.7\textwidth]{image2.png}}{goExample.mp4}
  \end{center}
\end{frame}
\section{Conclusion}
\begin{frame}{Conclusion}
  \begin{itemize}
    \item When working at a lower level, the RSA algorithm can be difficult to
      implement in a secure manner due to large bit length requirements of $n$,
      $e$, and $d$
      \pause
    \item Higher level languages can offer powerful libraries that allow for
      simple, practical RSA implementation
      \pause
    \item Though hardware-based cryptographic schemes appear to be the most
      exciting and future-proof alternatives, and will be the focus of my next
      presentation!
      \pause
    \item All code can be viewed on my Github, user name is: \textit{nigelgrey}
      \pause
    \item Along with our class slides, Christof Par and Jan Pelzl's
      \textit{Understanding Cryptography} served as my additional resource for
      this exploration
  \end{itemize}
\end{frame}
\begin{frame}{Conclusion}
  \begin{itemize}
      \item Thank You! What questions do you have?
      \end{itemize}
\end{frame}
\end{document}
